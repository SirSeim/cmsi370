\documentclass[journal,letterpaper]{article}
%sets the paper's header
\newcommand{\sethead}[6]{
    \lhead{\textit{#1}}
    \rhead{#2\\#3\\#4\\#5\\#6}
}

%Imports

\usepackage[utf8]{inputenc}
%headings
\usepackage{fancyhdr}
%double-space
\usepackage{setspace}
%variable margins
\usepackage{geometry}
%make the margins a bit less ridiculous
\newgeometry{top=1in,bottom=1.5in,left=1in,right=1in}
%images
\usepackage{graphicx}
\pagestyle{fancy}

\usepackage{tikz}
\usetikzlibrary{arrows}


\title{Interaction Design Usability Study}
\author{Edward Seim\thanks{group members Lauren K., Joshua K., and Adrian L.}}


\begin{document}
    \maketitle
    
    \section{Introduction}
    \label{Introduction}

    Our usability study focused on the reading experiences brought by iBooks and Kindle on the iOS platform.
    For our three metrics, we chose learnability, efficiency, and errors. Learnability and efficiency are measured in seconds, while satisfaction is measured on a scale of 1-10: 1 being completely unhappy, 10 being extremely satisfied. Our tests included:

    \begin{enumerate}
        \item Searching for and downloading a specific book
        \item Highlighting and making a note in the text
        \item Search for a specific quote in the book
    \end{enumerate}

    \section{Data}
    \label{Data}

    \subsection{iBooks Learnability}
    \begin{center}
        \centering
        \begin{tabular}{|l|l|l|l|l|l|l|l|}
            \hline
            \emph{Name} & \emph{Familiar?} & \emph{T1: Time} & \emph{T1: Errors} & \emph{T2: Time} & \emph{T2: Errors} & \emph{T3: Time} & \emph{T3: Errors} \\
            \hline
            Sam & No & 28 sec & 1 & 41 sec & 5 & 17 sec & 0 \\
            \hline
            Harris & No & 30 sec & 0 & 15 sec & 0 & 12 & 0 \\
            \hline
            Maurice & No & 35 sec & 1 & 21 sec & 0 & 15 sec & 0 \\
            \hline
            Albert & No & 22 sec & 1 & 180 sec & 8 & 24 sec & 0 \\
            \hline
            Ronald & No?? & 23 sec & 1 & 15 sec & 0 & 12 sec & 0 \\
            \hline
            Akers & No & 18 sec & 0 & 15 sec & 0 & 16 sec & 0 \\
            \hline
            Alex & No & 17 sec & 0 & 6 sec & 0 & 12 sec & 0 \\
            \hline
            Justin & No & 17 sec & 0 & 39 sec & 1 & 13 sec & 0 \\
            \hline
            Julia & No & 14 sec & 0 & 38 sec & 2 & 8 sec & 0 \\
            \hline
            Tori & No & 18 sec & 0 & 16 sec & 0 & 10 sec & 0 \\
            \hline
            Josh & No & 20 sec & 1 & 20 sec & 0 & 11 sec & 0 \\
            \hline
            Iko & No & 19 sec & 0 & 10 sec & 0 & 11 sec & 0 \\
            \hline
            Lauren & No & 10 sec & 0 & 9 sec & 1 & 17 sec & 0 \\
        \end{tabular}
        % \caption{This table shows learnability for iBooks}
    \end{center}

    \subsection{Kindle Learnability}
    \begin{center}
        \centering
        \begin{tabular}{|l|l|l|l|l|l|l|l|}
            \hline
            \emph{Name} & \emph{Familiar} & \emph{T1: Time} & \emph{T1: Errors} & \emph{T2: Time} & \emph{T2: Errors} & \emph{T3: Time} & \emph{T3: Errors} \\
            \hline
            Sam & No & 15 sec & 0 & 14 sec & 0 & 35 sec & 0 \\
            \hline
            Harris & No & 125 sec & 1 & 15 sec & 1 & 21 sec & 1 \\
            \hline
            Maurice & No & 34 sec & 0 & 19 sec & 0 & 39 sec & 2 \\
            \hline
            Albert & No & 20 sec & 0 & 17 sec & 1 & 28 sec & 1 \\
            \hline
            Ronald & No & 24 sec & 0 & 20 sec & 1 & 21 sec & 1 \\
            \hline
            Akers & No & 18 sec & 0 & 27 sec & 1 & 33 sec & 2 \\
            \hline
            Alex & No & 21 sec & 0 & 13 & 0 & 13 sec & 0 \\
            \hline
            Justin & No & 26 sec & 0 & 19 sec & 1 & 17 sec & 0 \\
            \hline
            Julia & No & 8 sec & 0 & 18 sec & 0 & 18 sec & 0 \\
            \hline
            Tori & No & 14 sec & 0 & 13 sec & 0 & 80 sec & 0 \\
            \hline
            Josh & No & 13 sec & 0 & 19 sec & 0 & 15 sec & 2 \\
            \hline
            Iko & No & 15 sec & 0 & 13 sec & 0 & 8 sec & 0 \\
            \hline
            Lauren & No & 19 sec & 0 & 6 sec & 0 & 15 sec & 7 \\
        \end{tabular}
        % \caption{This table shows learnability for Kindle}
    \end{center}

    \subsection{iBooks Efficiency}
    \begin{center}
        \centering
        \begin{tabular}{|l|l|l|l|l|l|l|l|}
            \hline
            \emph{Name} & \emph{Proficient} & \emph{T1: Time} & \emph{T1: Errors} & \emph{T2: Time} & \emph{T2: Errors} & \emph{T3: Time} & \emph{T3: Errors} \\
            \hline
            Josh & Yes & 11 sec & 0 & 7 sec & 0 & 8 sec & 0 \\
            \hline
            Ed & Yes & 6 sec & 0 & 7 sec & 1 & 6 sec & 0 \\
            \hline
            Adrian & Yes & 8 sec & 0 & 9 sec & 1 & 14 sec & 1 \\
            \hline
            Lauren & Yes & 7 sec & 0 & 10 sec & 1 & 8 sec & 0 \\
            \hline
            Julia & Yes & 10 sec & 1 & 9 sec & 0 & 7 sec & 0 \\
            \hline
            Tori & Yes & 6 sec & 0 & 5 sec & 0 & 4 sec & 0 \\
            \hline
            Harris & Yes & 9 sec & 0 & 6 sec & 0 & 9 sec & 0 \\
            \hline
            Justin & Yes & 15 sec & 1 & 25 sec & 1 & 15 sec & 0 \\
            \hline
            Iko & Yes & 13 sec & 0 & 8 sec & 0 & 8 sec & 0 \\
            \hline
            Akers & Yes & 8 sec & 0 & 10 sec & 12 sec & 0 \\
        \end{tabular}
        % \caption{This table shows efficiency for iBooks}
    \end{center}

    \subsection{Kindle Efficiency}
    \begin{center}
        \centering
        \begin{tabular}{|l|l|l|l|l|l|l|l|}
            \hline
            \emph{Name} & \emph{Proficient} & \emph{T1: Time} & \emph{T1: Errors} & \emph{T2: Time} & \emph{T2: Errors} & \emph{T3: Time} & \emph{T3: Errors} \\
            \hline
            Josh & Yes & 10 sec & 0 & 7 sec & 1 & 11 sec & 0 \\
            \hline
            Ed & Yes & 10 sec & 0 & 6 sec & 0 & 7 sec & 0 \\
            \hline
            Adrian & Yes & 12 sec & 0 & 10 sec & 2 & 11 sec & 0 \\
            \hline
            Lauren & Yes & 14 sec & 0 & 9 sec & 0 & 10 sec & 0 \\
            \hline
            Julia & Yes & 11 sec & 0 & 10 sec & 0 & 7 sec & 0 \\
            \hline
            Tori & Yes & 10 sec & 0 & 6 sec & 0 & 5 sec & 0 \\
            \hline
            Harris & Yes & 8 sec & 0 & 6 sec & 0 & 8 sec & 1 \\
            \hline
            Justin & Yes & 15 sec & 0 & 12 sec & 0 & 13 sec & 0 \\
            \hline
            Iko & Yes & 12 sec & 0 & 9 sec & 0 & 13 sec & 0 \\
            \hline
            Akers & Yes & 13 sec & 0 & 14 sec & 0 & 7 sec & 0 \\
        \end{tabular}
        % \caption{This table shows efficiency for Kindle}
    \end{center}

    \subsection{Learnability Averages}
    \begin{center}
        \centering
        \begin{tabular}{|l||l|l|}
            \hline
            \emph{Task} & \emph{iBooks} & \emph{Kindle} \\
            \hline
            Download & 20.846 sec & 27.846 sec \\
            \hline
            Highlight & 20.417 sec & 16.384 sec \\
            \hline
            Search & 13.692 sec & 26.307 sec \\
        \end{tabular}
        % \caption{This table shows learnability averages}
    \end{center}

    \subsection{Efficiency Averages}
    \begin{center}
        \centering
        \begin{tabular}{|l||l|l|}
            \hline
            \emph{Task} & \emph{iBooks} & \emph{Kindle} \\
            \hline
            Download & 9.3 sec & 11.5 sec \\
            \hline
            Highlight & 9.6 sec & 8.9 sec \\
            \hline
            Search & 9.1 sec & 8.7 sec \\
        \end{tabular}
        % \caption{This table shows efficiency averages}
    \end{center}

    \section{Analysis}
    \label{Analysis}

    \subsection{Learnability}
    In terms of learnability, iBooks had the quickest times. They were able to very quickly pick up the interfaces for downloading books, highlighting and making notes, and searching within the book. The only consistent complaint with iBooks was with the downloading task. Some users were seaching for the book and accidentally tried to download the whole collectio of books that book was a part of. For the task of highlighting and making a note, the only problem to come up for some users was the technique of highlighting. There was no readily available information while reading the book that makes it apparent that holding down on the text brings up the manifying glass and starts the selection. It you find yourself at the notes and highlights are where you would go to look at your past highlights and notes, if you don't have any you are given instructions on how to make a highlight which is a good use of the space than simply saying there are no highlights or notes. The other thing that confused users was the appearance of what looked like textual tools when in-text. This discouraged the attempt of using the iOS text selection gesture that appears throughout the system. Searching within the book had no remarkable errors or user complaints other than user typos. 

    Kindle had some interesting reponses with users. For the first task of downloading the book, the most important thing to note is that you cannot purchase and download a book from within the Kindle application on iOS. You must go out to the internet to purchase it. This was most users complaint about this task, as it meant they wuld have to go to the a webbrowser to purchase a book and then start reading it in the app. Various people tended to use either a google search to get to the book purchasing page or go to amazon.com and use their search function. The quicker results tended to be the users using a google search which can be done directly form the webbrowser without having to go to another site. For the second task of highlighting and making a note, the most common error was where the user would select some text and make a note on it and expect the selected text to also be highlighted. iBooks did conform to this expectation and many users took advtange of it. For the third task of searching the book, the most common error was that users were expecting the search to be live-updating. That is to say that the expectation was that as they would type into search, results would start coming up immediately. The Kindle app chose not to include this mechanic.

    By way of the quicker times and the fewer user errors, iBooks takes the trophy for better app in terms of learnability.

    \subsection{Efficiency}
    In terms of effiiency, both apps performed veyr well. iBooks tended to perform better in the download task. This can likely be associated with being able to purchase and download books within iBooks whereas one would have to purchase a book for Kindle from a webbrowser. For the tasks of highlighting and making a note, and searching within the book, neither app showed a significant advantage. There were also no major complaints about either app once the user had become proficient at using both apps. 





\end{document}



% \begin{document}
%     %ragged right with paragraph indents
%     \newlength{\saveparindent}
%     \setlength{\saveparindent}{\parindent}
%     \raggedright
%     \setlength{\parindent}{\saveparindent}
    
%     %%%%%%%%%%%%%%%%%%%%%%%%%%%%%%%%%%%%%%%%%%%%%%%%%%%%
%     %%     I   M   P   O   R   T   A   N   T   :      %%
%     %%  please remember to set the header to reflect  %%
%     %%      the class that you're writing for!        %%
%     %%%%%%%%%%%%%%%%%%%%%%%%%%%%%%%%%%%%%%%%%%%%%%%%%%%%
%     \sethead{Homework 4}{Joshua Kuroda, Jenny Miller}{Edward Seim, Adrian Lu}{CMSI 282}{Toal}{04/30/15}
    
%     %spacing
%     \doublespacing
%     \begin{center}
%         %necessary spacing between the header and the body
%     \end{center}
    
%     $ \: $
%     \begin{enumerate}
    
%     %1
%     \item
%     See table below.
%     \begin{table}[h]
%     \begin{tabular}{|c|c|c|c|c|c|c|c|c|}
%       \hline
%       \multicolumn{1}{|l|}{} & \multicolumn{8}{c|}{Iteration}                                \\ \hline
%       Node                   & 0        & 1        & 2        & 3        & 4  & 5  & 6  & 7  \\ \hline
%       A                      & $\infty$ & 7        & 7        & 7        & 7  & 7  & 7  & 7  \\ \cline{1-1}
%       B                      & $\infty$ & $\infty$ & 11       & 11       & 11 & 11 & 11 & 11 \\ \cline{1-1}
%       C                      & $\infty$ & 6        & 5        & 5        & 5  & 5  & 5  & 5  \\ \cline{1-1}
%       D                      & $\infty$ & $\infty$ & 8        & 7        & 7  & 7  & 7  & 7  \\ \cline{1-1}
%       E                      & $\infty$ & 6        & 6        & 6        & 6  & 6  & 6  & 6  \\ \cline{1-1}
%       F                      & $\infty$ & 5        & 4        & 4        & 4  & 4  & 4  & 4  \\ \cline{1-1}
%       G                      & $\infty$ & $\infty$ & $\infty$ & 2        & 1  & 1  & 1  & 1  \\ \cline{1-1}
%       H                      & $\infty$ & $\infty$ & 9        & 7        & 7  & 7  & 7  & 7  \\ \cline{1-1}
%       I                      & $\infty$ & $\infty$ & $\infty$ & $\infty$ & 8  & 7  & 7  & 7  \\ \cline{1-1}
%       S                      & 0        & 0        & 0        & 0        & 0  & 0  & 0  & 0  \\ \hline
%     \end{tabular}
%     \end{table}
    
%     %2
%     \item
    
%     Professor F. Lake is wrong! This is not a valid method. Adding an appropriate constant to each edge weight so that all weights become positive can change an inherent quality of the graph. Take this counter-example:
%     %counter-example goes here
%     \newline
%     \begin{tikzpicture}[->,>=stealth',shorten >=1pt,auto,node distance=3cm,
%   thick,main node/.style={circle,fill=green!20,draw,font=\sffamily\Large\bfseries}]

%     \node[main node] (T) {T};
%     \node[main node] (S) [left of=T] {S};

%     \path[every node/.style={font=\sffamily\small}]
%     (T) edge [bend right] node[above] {-1} (S)
%     (S) edge [bend right] node[below] {-1} (T);
%     \end{tikzpicture}

%     \begin{tikzpicture}[->,>=stealth',shorten >=1pt,auto,node distance=3cm,
%       thick,main node/.style={circle,fill=blue!20,draw,font=\sffamily\Large\bfseries}]

%       \node[main node] (T) {T};
%       \node[main node] (S) [left of=T] {S};

%       \path[every node/.style={font=\sffamily\small}]
%         (T) edge [bend right] node[above] {1} (S)
%         (S) edge [bend right] node[below] {1} (T);
%     \end{tikzpicture}
%     \newline
%     Suppose we added a constant 2 to each edge weight so all edges became positive, the shortest path from $s$ to $t$ in the new graph would be $s$ $\rightarrow$ $t$ with weight 1, using Dijkstra’s algorithm. However, in the original graph, the shortest path from $s$ to $t$ doesn’t exist. (Credit: CS Pomona)
%     %3
%     \item
    
%     By definition, Big-O of Dijkstra’s algorithm $O((|V| + |E|) log W )$.  Using Dial's implementation, we have a maximum edge weight of W and a vertex can be updated at most $|V| - 1$ times, we get a bound of $O(W|V|)$. Reincorporating this into Dijkstra's algorithm, we achieve a bound of $O(W |V | + |E|)$.
    
%     %4
%     \item
%       \begin{enumerate}
%       \item
%       See table below.
%         \begin{table}[h]
%         \begin{tabular}{ccccccccc}
%         \multicolumn{9}{c}{Intermediate Values of the Cost Array}                                                                                                                                                                                                                                                            \\ \hline
%         \multicolumn{1}{|c|}{Set S}                   & \multicolumn{1}{c|}{A} & \multicolumn{1}{c|}{B}          & \multicolumn{1}{c|}{C}          & \multicolumn{1}{c|}{D}          & \multicolumn{1}{c|}{E}          & \multicolumn{1}{c|}{F}          & \multicolumn{1}{c|}{G}          & \multicolumn{1}{c|}{H}          \\ \hline
%         \multicolumn{1}{|c|}{\{A\}}                   & \multicolumn{1}{c|}{0} & \multicolumn{1}{c|}{\textbf{1}} & \multicolumn{1}{c|}{$\infty$}   & \multicolumn{1}{c|}{$\infty$}   & \multicolumn{1}{c|}{4}          & \multicolumn{1}{c|}{8}          & \multicolumn{1}{c|}{$\infty$}   & \multicolumn{1}{c|}{$\infty$}   \\ \hline
%         \multicolumn{1}{|c|}{\{A, B\}}                & \multicolumn{1}{c|}{}  & \multicolumn{1}{c|}{}           & \multicolumn{1}{c|}{\textbf{2}} & \multicolumn{1}{c|}{$\infty$}   & \multicolumn{1}{c|}{4}          & \multicolumn{1}{c|}{6}          & \multicolumn{1}{c|}{6}          & \multicolumn{1}{c|}{$\infty$}   \\ \hline
%         \multicolumn{1}{|c|}{\{A, B, C\}}             & \multicolumn{1}{c|}{}  & \multicolumn{1}{c|}{}           & \multicolumn{1}{c|}{}           & \multicolumn{1}{c|}{3}          & \multicolumn{1}{c|}{4}          & \multicolumn{1}{c|}{6}          & \multicolumn{1}{c|}{\textbf{2}} & \multicolumn{1}{c|}{$\infty$}   \\ \hline
%         \multicolumn{1}{|c|}{\{A, B, C, G\}}          & \multicolumn{1}{c|}{}  & \multicolumn{1}{c|}{}           & \multicolumn{1}{c|}{}           & \multicolumn{1}{c|}{\textbf{1}} & \multicolumn{1}{c|}{4}          & \multicolumn{1}{c|}{1}          & \multicolumn{1}{c|}{}           & \multicolumn{1}{c|}{1}          \\ \hline
%         \multicolumn{1}{|c|}{\{A, B, C, G, D\}}       & \multicolumn{1}{c|}{}  & \multicolumn{1}{c|}{}           & \multicolumn{1}{c|}{}           & \multicolumn{1}{c|}{}           & \multicolumn{1}{c|}{4}          & \multicolumn{1}{c|}{\textbf{1}} & \multicolumn{1}{c|}{}           & \multicolumn{1}{c|}{1}          \\ \hline
%         \multicolumn{1}{|c|}{\{A, B, C, G, D, F\}}    & \multicolumn{1}{c|}{}  & \multicolumn{1}{c|}{}           & \multicolumn{1}{c|}{}           & \multicolumn{1}{c|}{}           & \multicolumn{1}{c|}{4}          & \multicolumn{1}{c|}{}           & \multicolumn{1}{c|}{}           & \multicolumn{1}{c|}{\textbf{1}} \\ \hline
%         \multicolumn{1}{|c|}{\{A, B, C, G, D, F, H\}} & \multicolumn{1}{c|}{}  & \multicolumn{1}{c|}{}           & \multicolumn{1}{c|}{}           & \multicolumn{1}{c|}{}           & \multicolumn{1}{c|}{\textbf{4}} & \multicolumn{1}{c|}{}           & \multicolumn{1}{c|}{}           & \multicolumn{1}{c|}{}           \\ \hline
%         \end{tabular}
%         \end{table}
      
%       \item
%         Running Kruskal's Algorithm requires us to first order the edges in the graph by increasing weight ((x, y): w means that the edge (x, y) has weight w): 
%         \newline
%         \{(A, B): 1, (D, G): 1, (F, G): 1, (H, G): 1, (B, C): 2, (C, G): 2, (C, D): 3, (A,~E): 4, (D, H): 4, (E, F): 5, (B, F): 6, (B, G): 6, (A, F): 8\}
%         \newline
%         Now loop over the ordered edges:
%         \newline
%         (A, B) \newline
%         data structure: ({$A^{0}$, $B^{1}$}, {A$\rightarrow$B}) \newline

%         (D, G) \newline
%         data structure: ({$A^{0}$, $B^{1}$, $D^{0}$, $G^{1}$}, {A$\rightarrow$B, D$\rightarrow$G}) \newline
        
%         (F, G) \newline
%         data structure: ({$A^{0}$, $B^{1}$, $D^{0}$, $F^{0}$, $G^{1}$}, {A$\rightarrow$B, D$\rightarrow$G, F$\rightarrow$G}) \newline

%         (H, G) \newline
%         data structure: ({$A^{0}$, $B^{1}$, $D^{0}$, $F^{0}$, $H^{0}$, $G^{1}$}, {A$\rightarrow$B, D$\rightarrow$G, F$\rightarrow$G, H$\rightarrow$G}) \newline

%         (B, C) \newline
%         data structure: ({$A^{0}$, $C^{0}$, $B^{1}$, $D^{0}$, $F^{0}$, $H^{0}$, $G^{1}$}, {A$\rightarrow$B, C$\rightarrow$B, D$\rightarrow$G, F$\rightarrow$G, H$\rightarrow$G}) \newline

%         (C, G) \newline
%         data structure: ({$A^{0}$, $C^{0}$, $B^{1}$, $D^{0}$, $F^{0}$, $H^{0}$, $G^{2}$}, {A$\rightarrow$B, C$\rightarrow$B, D$\rightarrow$G, F$\rightarrow$G, H$\rightarrow$G, B$\rightarrow$G}) \newline

%         (C, D) \newline
%         data structure: ({$A^{0}$, $C^{0}$, $D^{0}$, $F^{0}$, $H^{0}$, $B^{1}$, $G^{2}$}, {A$\rightarrow$B, C$\rightarrow$G, D$\rightarrow$G, F$\rightarrow$G, H$\rightarrow$G, B$\rightarrow$G}) \newline

%         (A, E) \newline
%         data structure: ({$A^{0}$, $C^{0}$, $D^{0}$, $F^{0}$, $H^{0}$, $E^{0}$, $B^{1}$, $G^{2}$}, {A$\rightarrow$B, C$\rightarrow$G, D$\rightarrow$G, F$\rightarrow$G, H$\rightarrow$G, B$\rightarrow$G, E$\rightarrow$G}) \newline

%         (D, H), (E, F), (B, F), (B, G), (A, F) \newline
%         data structure: ({$A^{0}$, $C^{0}$, $D^{0}$, $F^{0}$, $H^{0}$, $E^{0}$, $B^{1}$, $G^{2}$}, {A$\rightarrow$G, C$\rightarrow$G, D$\rightarrow$G, F$\rightarrow$G, H$\rightarrow$G, B$\rightarrow$G, E$\rightarrow$G}) \newline
%       \end{enumerate}
%     %5
%     \item 
%         Function is in file sum\_checker.py
    
%     %6
%     \item
%         Solver is in file KirkmanSolver.java
        
%     \end{enumerate}
% \end{document}








% %basic graph with nodes and arrows:
% \begin{tikzpicture}[->,>=stealth',shorten >=1pt,auto,node distance=2cm,
%     thin,main node/.style={circle,fill=blue!20,draw,font=\sffamily\Large\bfseries}]

% \node[main node] (1) {1};
% \node[main node] (2) [right of=1] {2};
% \node[main node] (3) [right of=2] {3};
% \node[main node] (4) [right of=3] {4};
% \node[main node] (5) [below of=1] {5};
% \node[main node] (6) [below of=2] {6};
% \node[main node] (7) [below of=3] {7};
% \node[main node] (8) [below of=4] {8};

% \path[every node/.style={font=\sffamily\small}]
% (1) edge [bend right] (2)
% (2) edge [bend right] (3)
% (1) edge [bend left=60] (8);

% \end{tikzpicture}