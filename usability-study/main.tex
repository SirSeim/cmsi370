\documentclass[journal,letterpaper]{article}
%sets the paper's header
\newcommand{\sethead}[6]{
    \lhead{\textit{#1}}
    \rhead{#2\\#3\\#4\\#5\\#6}
}

%Imports

\usepackage[utf8]{inputenc}
%headings
\usepackage{fancyhdr}
%double-space
\usepackage{setspace}
%variable margins
\usepackage{geometry}
%make the margins a bit less ridiculous
\newgeometry{top=1in,bottom=1.5in,left=1in,right=1in}
%images
\usepackage{graphicx}
\pagestyle{fancy}

\usepackage{tikz}
\usetikzlibrary{arrows}


\title{Interaction Design Usability Study}
\author{Edward Seim\thanks{group members Lauren K., Joshua K., and Adrian L.}}


\begin{document}
    \maketitle
    
    \section{Introduction}
    \label{Introduction}

    Our usability study focused on the reading experiences brought by iBooks and Kindle on the iOS platform.
    For our three metrics, we chose learnability, efficiency, and errors. Learnability and efficiency are measured in seconds, while satisfaction is measured on a scale of 1-10: 1 being completely unhappy, 10 being extremely satisfied. Our tests included:

    \begin{enumerate}
        \item Searching for and downloading a specific book
        \item Highlighting and making a note in the text
        \item Search for a specific quote in the book
    \end{enumerate}

    \section{Data}
    \label{Data}

    \subsection{iBooks Learnability}
    \begin{center}
        \centering
        \begin{tabular}{|l|l|l|l|l|l|l|l|}
            \hline
            \emph{Name} & \emph{Familiar?} & \emph{T1: Time} & \emph{T1: Errors} & \emph{T2: Time} & \emph{T2: Errors} & \emph{T3: Time} & \emph{T3: Errors} \\
            \hline
            Sam & No & 28 sec & 1 & 41 sec & 5 & 17 sec & 0 \\
            \hline
            Harris & No & 30 sec & 0 & 15 sec & 0 & 12 & 0 \\
            \hline
            Maurice & No & 35 sec & 1 & 21 sec & 0 & 15 sec & 0 \\
            \hline
            Albert & No & 22 sec & 1 & 180 sec & 8 & 24 sec & 0 \\
            \hline
            Ronald & No?? & 23 sec & 1 & 15 sec & 0 & 12 sec & 0 \\
            \hline
            Akers & No & 18 sec & 0 & 15 sec & 0 & 16 sec & 0 \\
            \hline
            Alex & No & 17 sec & 0 & 6 sec & 0 & 12 sec & 0 \\
            \hline
            Justin & No & 17 sec & 0 & 39 sec & 1 & 13 sec & 0 \\
            \hline
            Julia & No & 14 sec & 0 & 38 sec & 2 & 8 sec & 0 \\
            \hline
            Tori & No & 18 sec & 0 & 16 sec & 0 & 10 sec & 0 \\
            \hline
            Josh & No & 20 sec & 1 & 20 sec & 0 & 11 sec & 0 \\
            \hline
            Iko & No & 19 sec & 0 & 10 sec & 0 & 11 sec & 0 \\
            \hline
            Lauren & No & 10 sec & 0 & 9 sec & 1 & 17 sec & 0 \\
        \end{tabular}
        % \caption{This table shows learnability for iBooks}
    \end{center}

    \subsection{Kindle Learnability}
    \begin{center}
        \centering
        \begin{tabular}{|l|l|l|l|l|l|l|l|}
            \hline
            \emph{Name} & \emph{Familiar} & \emph{T1: Time} & \emph{T1: Errors} & \emph{T2: Time} & \emph{T2: Errors} & \emph{T3: Time} & \emph{T3: Errors} \\
            \hline
            Sam & No & 15 sec & 0 & 14 sec & 0 & 35 sec & 0 \\
            \hline
            Harris & No & 125 sec & 1 & 15 sec & 1 & 21 sec & 1 \\
            \hline
            Maurice & No & 34 sec & 0 & 19 sec & 0 & 39 sec & 2 \\
            \hline
            Albert & No & 20 sec & 0 & 17 sec & 1 & 28 sec & 1 \\
            \hline
            Ronald & No & 24 sec & 0 & 20 sec & 1 & 21 sec & 1 \\
            \hline
            Akers & No & 18 sec & 0 & 27 sec & 1 & 33 sec & 2 \\
            \hline
            Alex & No & 21 sec & 0 & 13 & 0 & 13 sec & 0 \\
            \hline
            Justin & No & 26 sec & 0 & 19 sec & 1 & 17 sec & 0 \\
            \hline
            Julia & No & 8 sec & 0 & 18 sec & 0 & 18 sec & 0 \\
            \hline
            Tori & No & 14 sec & 0 & 13 sec & 0 & 80 sec & 0 \\
            \hline
            Josh & No & 13 sec & 0 & 19 sec & 0 & 15 sec & 2 \\
            \hline
            Iko & No & 15 sec & 0 & 13 sec & 0 & 8 sec & 0 \\
            \hline
            Lauren & No & 19 sec & 0 & 6 sec & 0 & 15 sec & 7 \\
        \end{tabular}
        % \caption{This table shows learnability for Kindle}
    \end{center}

    \subsection{iBooks Efficiency}
    \begin{center}
        \centering
        \begin{tabular}{|l|l|l|l|l|l|l|l|}
            \hline
            \emph{Name} & \emph{Proficient} & \emph{T1: Time} & \emph{T1: Errors} & \emph{T2: Time} & \emph{T2: Errors} & \emph{T3: Time} & \emph{T3: Errors} \\
            \hline
            Josh & Yes & 11 sec & 0 & 7 sec & 0 & 8 sec & 0 \\
            \hline
            Ed & Yes & 6 sec & 0 & 7 sec & 1 & 6 sec & 0 \\
            \hline
            Adrian & Yes & 8 sec & 0 & 9 sec & 1 & 14 sec & 1 \\
            \hline
            Lauren & Yes & 7 sec & 0 & 10 sec & 1 & 8 sec & 0 \\
            \hline
            Julia & Yes & 10 sec & 1 & 9 sec & 0 & 7 sec & 0 \\
            \hline
            Tori & Yes & 6 sec & 0 & 5 sec & 0 & 4 sec & 0 \\
            \hline
            Harris & Yes & 9 sec & 0 & 6 sec & 0 & 9 sec & 0 \\
            \hline
            Justin & Yes & 15 sec & 1 & 25 sec & 1 & 15 sec & 0 \\
            \hline
            Iko & Yes & 13 sec & 0 & 8 sec & 0 & 8 sec & 0 \\
            \hline
            Akers & Yes & 8 sec & 0 & 10 sec & 12 sec & 0 \\
        \end{tabular}
        % \caption{This table shows efficiency for iBooks}
    \end{center}

    \subsection{Kindle Efficiency}
    \begin{center}
        \centering
        \begin{tabular}{|l|l|l|l|l|l|l|l|}
            \hline
            \emph{Name} & \emph{Proficient} & \emph{T1: Time} & \emph{T1: Errors} & \emph{T2: Time} & \emph{T2: Errors} & \emph{T3: Time} & \emph{T3: Errors} \\
            \hline
            Josh & Yes & 10 sec & 0 & 7 sec & 1 & 11 sec & 0 \\
            \hline
            Ed & Yes & 10 sec & 0 & 6 sec & 0 & 7 sec & 0 \\
            \hline
            Adrian & Yes & 12 sec & 0 & 10 sec & 2 & 11 sec & 0 \\
            \hline
            Lauren & Yes & 14 sec & 0 & 9 sec & 0 & 10 sec & 0 \\
            \hline
            Julia & Yes & 11 sec & 0 & 10 sec & 0 & 7 sec & 0 \\
            \hline
            Tori & Yes & 10 sec & 0 & 6 sec & 0 & 5 sec & 0 \\
            \hline
            Harris & Yes & 8 sec & 0 & 6 sec & 0 & 8 sec & 1 \\
            \hline
            Justin & Yes & 15 sec & 0 & 12 sec & 0 & 13 sec & 0 \\
            \hline
            Iko & Yes & 12 sec & 0 & 9 sec & 0 & 13 sec & 0 \\
            \hline
            Akers & Yes & 13 sec & 0 & 14 sec & 0 & 7 sec & 0 \\
        \end{tabular}
        % \caption{This table shows efficiency for Kindle}
    \end{center}

    \subsection{Learnability Averages}
    \begin{center}
        \centering
        \begin{tabular}{|l||l|l|}
            \hline
            \emph{Task} & \emph{iBooks} & \emph{Kindle} \\
            \hline
            Download & 20.846 sec & 27.846 sec \\
            \hline
            Highlight & 20.417 sec & 16.384 sec \\
            \hline
            Search & 13.692 sec & 26.307 sec \\
        \end{tabular}
        % \caption{This table shows learnability averages}
    \end{center}

    \subsection{Efficiency Averages}
    \begin{center}
        \centering
        \begin{tabular}{|l||l|l|}
            \hline
            \emph{Task} & \emph{iBooks} & \emph{Kindle} \\
            \hline
            Download & 9.3 sec & 11.5 sec \\
            \hline
            Highlight & 9.6 sec & 8.9 sec \\
            \hline
            Search & 9.1 sec & 8.7 sec \\
        \end{tabular}
        % \caption{This table shows efficiency averages}
    \end{center}

    \section{Analysis}
    \label{Analysis}

    \subsection{Learnability}
    In terms of learnability, iBooks had the quickest times. They were able to very quickly pick up the interfaces for downloading books, highlighting and making notes, and searching within the book. The only consistent complaint with iBooks was with the downloading task. Some users were seaching for the book and accidentally tried to download the whole collectio of books that book was a part of. For the task of highlighting and making a note, the only problem to come up for some users was the technique of highlighting. There was no readily available information while reading the book that makes it apparent that holding down on the text brings up the manifying glass and starts the selection. It you find yourself at the notes and highlights are where you would go to look at your past highlights and notes, if you don't have any you are given instructions on how to make a highlight which is a good use of the space than simply saying there are no highlights or notes. The other thing that confused users was the appearance of what looked like textual tools when in-text. This discouraged the attempt of using the iOS text selection gesture that appears throughout the system. Searching within the book had no remarkable errors or user complaints other than user typos. 

    Kindle had some interesting reponses with users. For the first task of downloading the book, the most important thing to note is that you cannot purchase and download a book from within the Kindle application on iOS. You must go out to the internet to purchase it. This was most users complaint about this task, as it meant they wuld have to go to the a webbrowser to purchase a book and then start reading it in the app. Various people tended to use either a google search to get to the book purchasing page or go to amazon.com and use their search function. The quicker results tended to be the users using a google search which can be done directly form the webbrowser without having to go to another site. For the second task of highlighting and making a note, the most common error was where the user would select some text and make a note on it and expect the selected text to also be highlighted. iBooks did conform to this expectation and many users took advtange of it. For the third task of searching the book, the most common error was that users were expecting the search to be live-updating. That is to say that the expectation was that as they would type into search, results would start coming up immediately. The Kindle app chose not to include this mechanic.

    By way of the quicker times and the fewer user errors, iBooks takes the trophy for better app in terms of learnability.

    \subsection{Efficiency}
    In terms of effiiency, both apps performed veyr well. iBooks tended to perform better in the download task. This can likely be associated with being able to purchase and download books within iBooks whereas one would have to purchase a book for Kindle from a webbrowser. For the tasks of highlighting and making a note, and searching within the book, neither app showed a significant advantage. There were also no major complaints about either app once the user had become proficient at using both apps. 

    \section{Heuristic Evaluation}
    \label{Hueristic Evaluation}

    Before diving into the specific guidlines provided for Apple and Amazon experiences, it is important to remember that on iOS, Apple has their own documents about how it should work in iOS, whereas Amazon's documents compete against Apple's platform. This means Amazon has to make choices between using their own documents, or adopting Apple's, or some combination of the two.

    \subsection{iOS Human Interface Guidelines}
    Apple goes very indepth with their guidelines, but with a big picture of some iOS themes. Starting with the themes, Apple lists the major themes of iOS to be Deference, Clarity, and Depth. According to Apple, Deference is having a user interface that helps people understand and interact with the content without competing against it. Whenever having the option of having a bezel or gradient or dropshadow, it is discouraged. The prime focus is to be the content and the UI to support it. Things like translucence hint at content behind and allow for the user to get at that content. Clarity is another way Apple suggests to ensuring the importance of content in an app. Some of the ways they suggest to ensure this is by providing plenty of negative space, use of color, using system fonts, and borderless buttons. The effects of negative space are two fold, making things easy to understand, and making the app looked more focused. Color is important because using a consitent color in your app means that its easy to highlight things to be interacted with and yet keeps the whole app clean. Borderless buttons follow the same standard of using colors. They go further to include call-to-action titles to indicate interactivity. The last theme of depth is a tool to protray a sense of hierarchy and position, to easily understand relationships between onscreen objects. Zooming transistions emphasize the hierachy of the content. For example, in calendar, you have zooming from the year view into the month view when you tap on a specific month. This enforces the hierarchy of months in years. 

    \subsection{Amazon User Experience Guidelines}
    Amazon is a bit more sparce in their guidlines as opposed to Apple's. However, they do share some characteristics. Amazon believes in the world not being flat. This is analogous to focus Apple places on the use of depth to demonstrate hierachy. They also believe in content focus. The idea is to make the content the focus and the primary user interface. Amazon also places a focus on the simplicity of the interface. This makes a design crisp, precise, and efficient. It also means details and are readily identifiable. 

    \subsection{iBooks and Apple Guidelines}
    Not very surprisingly, Apple did an excellent job of following their guidelines. Starting with the first theme of Deference, we immediately see in iBooks that while reading a book, the text is front and center. Typically while reading, all the controls are gone and the main interaction is a natural page flip to continue reading. When the controls are brought up, they are non-intrusive to the content. Looking at Clarity, reading a book has a lot of negative space surrounding the content while in normal reading. Part of this negative space is taken up when the controls are brought up. However, there is still enough negative space to maintain the focus on the content and make the controls noticable. While reading, the controls in-text are all brown, differenciated from the might lighter shade of brown for the background. This infers their interactability, consistent with Apple's guidelines. As different people find certain find certain fonts easier to read for extended periods than others, Apple chose not to enforce their system font, however all their font options take legibility seriously. In the spirit of maintaining a healthy amount of negative space, all the buttons are borderless and minimalistic. To maintain depth, iBooks animates transitions between tasks with emphasis on opening and closing books. Opening a book has the book zoom in on the cover and open to your last page. Closing the book animates the closing and placement back into the list. This reenforces the hierachy of having books which can be individually interacted with. When performing a task like typing a note on a selection made, the page slides up on top of the text and once you've made your note, it slides bac down revealing the text once more. This preserves the hierachy of text being the main focus and the act of making a note to be a task to be completed and return to the reading. In these ways, iBooks closely follows the Apple design guidelines. 

    \subsection{Kindle and Amazon Guidelines}
    The problem with Amazon's guidelines is that their vagueness means one could argue both for and against a particular element following the guidelines. While reading, the focus is clearly on the content. It takes up the full screen and is therefore the primary task. However, by being fullscreen, it doesn't have the negative space that provides the appearance of simplicity, every page seemingly very busy. It does take advtange of depth. When you want to pull up the controls, the buttons are overlayed above and below the text, with a transition that makes it clear the text it below, simply temporarily covered up. The menu button, which contains all other controls not visible on the over-text controls, overlays onto the text, darkening it while still showing it. This preserves the hierarchy of the text being the focus and this menu being an action that goes away to go back to focusing on the text. In these ways, the Kindle app for iOS does follow the Amazon guidelines.

    \section{Conclusion}
    \label{Conclusion}

    With the metrics of learnability, efficiency, and errors, we have compared iBooks and Kindle in their abilities to allow the user to effectively read their book and perform tasks concerning that book. It appears that iBooks has succeeded more than Kindle in this goal. iBooks produced the least number of user errors, the quicker learnability times, and performed equally for efficiency. The similarities between iBooks and the rest of iOS means that most new users will already be familiar with the basic gestures and not require many new skills. Kindle on iOS has the disadvantage of not being able to do things their own way. This is most evident with the purchasing within Safari feature. Kindle had a lot going against it and it seems they were unable to overcome Apple's advantage.


\end{document}
