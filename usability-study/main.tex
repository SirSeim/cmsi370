\documentclass[12pt,letterpaper]{article}
%sets the paper's header
\newcommand{\sethead}[6]{
    \lhead{\textit{#1}}
    \rhead{#2\\#3\\#4\\#5\\#6}
}

%Imports

\usepackage[utf8]{inputenc}
%headings
\usepackage{fancyhdr}
%double-space
\usepackage{setspace}
%variable margins
\usepackage{geometry}
%make the margins a bit less ridiculous
\newgeometry{top=1in,bottom=1.5in,left=1in,right=1in}
%images
\usepackage{graphicx}
\pagestyle{fancy}

\usepackage{tikz}
\usetikzlibrary{arrows}


\title{Interaction Design Usability Study}
\author{Edward Seim (and group members Lauren K., Joshua K., and Adrian Lu)}

\begin{document}
    \maketitle
    
    \section{Introduction}
    \label{Introduction}

    Our usability study focused on the reading experiences brought by iBooks and Kindle on the iOS platform. 
\end{document}



% \begin{document}
%     %ragged right with paragraph indents
%     \newlength{\saveparindent}
%     \setlength{\saveparindent}{\parindent}
%     \raggedright
%     \setlength{\parindent}{\saveparindent}
    
%     %%%%%%%%%%%%%%%%%%%%%%%%%%%%%%%%%%%%%%%%%%%%%%%%%%%%
%     %%     I   M   P   O   R   T   A   N   T   :      %%
%     %%  please remember to set the header to reflect  %%
%     %%      the class that you're writing for!        %%
%     %%%%%%%%%%%%%%%%%%%%%%%%%%%%%%%%%%%%%%%%%%%%%%%%%%%%
%     \sethead{Homework 4}{Joshua Kuroda, Jenny Miller}{Edward Seim, Adrian Lu}{CMSI 282}{Toal}{04/30/15}
    
%     %spacing
%     \doublespacing
%     \begin{center}
%         %necessary spacing between the header and the body
%     \end{center}
    
%     $ \: $
%     \begin{enumerate}
    
%     %1
%     \item
%     See table below.
%     \begin{table}[h]
%     \begin{tabular}{|c|c|c|c|c|c|c|c|c|}
%       \hline
%       \multicolumn{1}{|l|}{} & \multicolumn{8}{c|}{Iteration}                                \\ \hline
%       Node                   & 0        & 1        & 2        & 3        & 4  & 5  & 6  & 7  \\ \hline
%       A                      & $\infty$ & 7        & 7        & 7        & 7  & 7  & 7  & 7  \\ \cline{1-1}
%       B                      & $\infty$ & $\infty$ & 11       & 11       & 11 & 11 & 11 & 11 \\ \cline{1-1}
%       C                      & $\infty$ & 6        & 5        & 5        & 5  & 5  & 5  & 5  \\ \cline{1-1}
%       D                      & $\infty$ & $\infty$ & 8        & 7        & 7  & 7  & 7  & 7  \\ \cline{1-1}
%       E                      & $\infty$ & 6        & 6        & 6        & 6  & 6  & 6  & 6  \\ \cline{1-1}
%       F                      & $\infty$ & 5        & 4        & 4        & 4  & 4  & 4  & 4  \\ \cline{1-1}
%       G                      & $\infty$ & $\infty$ & $\infty$ & 2        & 1  & 1  & 1  & 1  \\ \cline{1-1}
%       H                      & $\infty$ & $\infty$ & 9        & 7        & 7  & 7  & 7  & 7  \\ \cline{1-1}
%       I                      & $\infty$ & $\infty$ & $\infty$ & $\infty$ & 8  & 7  & 7  & 7  \\ \cline{1-1}
%       S                      & 0        & 0        & 0        & 0        & 0  & 0  & 0  & 0  \\ \hline
%     \end{tabular}
%     \end{table}
    
%     %2
%     \item
    
%     Professor F. Lake is wrong! This is not a valid method. Adding an appropriate constant to each edge weight so that all weights become positive can change an inherent quality of the graph. Take this counter-example:
%     %counter-example goes here
%     \newline
%     \begin{tikzpicture}[->,>=stealth',shorten >=1pt,auto,node distance=3cm,
%   thick,main node/.style={circle,fill=green!20,draw,font=\sffamily\Large\bfseries}]

%     \node[main node] (T) {T};
%     \node[main node] (S) [left of=T] {S};

%     \path[every node/.style={font=\sffamily\small}]
%     (T) edge [bend right] node[above] {-1} (S)
%     (S) edge [bend right] node[below] {-1} (T);
%     \end{tikzpicture}

%     \begin{tikzpicture}[->,>=stealth',shorten >=1pt,auto,node distance=3cm,
%       thick,main node/.style={circle,fill=blue!20,draw,font=\sffamily\Large\bfseries}]

%       \node[main node] (T) {T};
%       \node[main node] (S) [left of=T] {S};

%       \path[every node/.style={font=\sffamily\small}]
%         (T) edge [bend right] node[above] {1} (S)
%         (S) edge [bend right] node[below] {1} (T);
%     \end{tikzpicture}
%     \newline
%     Suppose we added a constant 2 to each edge weight so all edges became positive, the shortest path from $s$ to $t$ in the new graph would be $s$ $\rightarrow$ $t$ with weight 1, using Dijkstra’s algorithm. However, in the original graph, the shortest path from $s$ to $t$ doesn’t exist. (Credit: CS Pomona)
%     %3
%     \item
    
%     By definition, Big-O of Dijkstra’s algorithm $O((|V| + |E|) log W )$.  Using Dial's implementation, we have a maximum edge weight of W and a vertex can be updated at most $|V| - 1$ times, we get a bound of $O(W|V|)$. Reincorporating this into Dijkstra's algorithm, we achieve a bound of $O(W |V | + |E|)$.
    
%     %4
%     \item
%       \begin{enumerate}
%       \item
%       See table below.
%         \begin{table}[h]
%         \begin{tabular}{ccccccccc}
%         \multicolumn{9}{c}{Intermediate Values of the Cost Array}                                                                                                                                                                                                                                                            \\ \hline
%         \multicolumn{1}{|c|}{Set S}                   & \multicolumn{1}{c|}{A} & \multicolumn{1}{c|}{B}          & \multicolumn{1}{c|}{C}          & \multicolumn{1}{c|}{D}          & \multicolumn{1}{c|}{E}          & \multicolumn{1}{c|}{F}          & \multicolumn{1}{c|}{G}          & \multicolumn{1}{c|}{H}          \\ \hline
%         \multicolumn{1}{|c|}{\{A\}}                   & \multicolumn{1}{c|}{0} & \multicolumn{1}{c|}{\textbf{1}} & \multicolumn{1}{c|}{$\infty$}   & \multicolumn{1}{c|}{$\infty$}   & \multicolumn{1}{c|}{4}          & \multicolumn{1}{c|}{8}          & \multicolumn{1}{c|}{$\infty$}   & \multicolumn{1}{c|}{$\infty$}   \\ \hline
%         \multicolumn{1}{|c|}{\{A, B\}}                & \multicolumn{1}{c|}{}  & \multicolumn{1}{c|}{}           & \multicolumn{1}{c|}{\textbf{2}} & \multicolumn{1}{c|}{$\infty$}   & \multicolumn{1}{c|}{4}          & \multicolumn{1}{c|}{6}          & \multicolumn{1}{c|}{6}          & \multicolumn{1}{c|}{$\infty$}   \\ \hline
%         \multicolumn{1}{|c|}{\{A, B, C\}}             & \multicolumn{1}{c|}{}  & \multicolumn{1}{c|}{}           & \multicolumn{1}{c|}{}           & \multicolumn{1}{c|}{3}          & \multicolumn{1}{c|}{4}          & \multicolumn{1}{c|}{6}          & \multicolumn{1}{c|}{\textbf{2}} & \multicolumn{1}{c|}{$\infty$}   \\ \hline
%         \multicolumn{1}{|c|}{\{A, B, C, G\}}          & \multicolumn{1}{c|}{}  & \multicolumn{1}{c|}{}           & \multicolumn{1}{c|}{}           & \multicolumn{1}{c|}{\textbf{1}} & \multicolumn{1}{c|}{4}          & \multicolumn{1}{c|}{1}          & \multicolumn{1}{c|}{}           & \multicolumn{1}{c|}{1}          \\ \hline
%         \multicolumn{1}{|c|}{\{A, B, C, G, D\}}       & \multicolumn{1}{c|}{}  & \multicolumn{1}{c|}{}           & \multicolumn{1}{c|}{}           & \multicolumn{1}{c|}{}           & \multicolumn{1}{c|}{4}          & \multicolumn{1}{c|}{\textbf{1}} & \multicolumn{1}{c|}{}           & \multicolumn{1}{c|}{1}          \\ \hline
%         \multicolumn{1}{|c|}{\{A, B, C, G, D, F\}}    & \multicolumn{1}{c|}{}  & \multicolumn{1}{c|}{}           & \multicolumn{1}{c|}{}           & \multicolumn{1}{c|}{}           & \multicolumn{1}{c|}{4}          & \multicolumn{1}{c|}{}           & \multicolumn{1}{c|}{}           & \multicolumn{1}{c|}{\textbf{1}} \\ \hline
%         \multicolumn{1}{|c|}{\{A, B, C, G, D, F, H\}} & \multicolumn{1}{c|}{}  & \multicolumn{1}{c|}{}           & \multicolumn{1}{c|}{}           & \multicolumn{1}{c|}{}           & \multicolumn{1}{c|}{\textbf{4}} & \multicolumn{1}{c|}{}           & \multicolumn{1}{c|}{}           & \multicolumn{1}{c|}{}           \\ \hline
%         \end{tabular}
%         \end{table}
      
%       \item
%         Running Kruskal's Algorithm requires us to first order the edges in the graph by increasing weight ((x, y): w means that the edge (x, y) has weight w): 
%         \newline
%         \{(A, B): 1, (D, G): 1, (F, G): 1, (H, G): 1, (B, C): 2, (C, G): 2, (C, D): 3, (A,~E): 4, (D, H): 4, (E, F): 5, (B, F): 6, (B, G): 6, (A, F): 8\}
%         \newline
%         Now loop over the ordered edges:
%         \newline
%         (A, B) \newline
%         data structure: ({$A^{0}$, $B^{1}$}, {A$\rightarrow$B}) \newline

%         (D, G) \newline
%         data structure: ({$A^{0}$, $B^{1}$, $D^{0}$, $G^{1}$}, {A$\rightarrow$B, D$\rightarrow$G}) \newline
        
%         (F, G) \newline
%         data structure: ({$A^{0}$, $B^{1}$, $D^{0}$, $F^{0}$, $G^{1}$}, {A$\rightarrow$B, D$\rightarrow$G, F$\rightarrow$G}) \newline

%         (H, G) \newline
%         data structure: ({$A^{0}$, $B^{1}$, $D^{0}$, $F^{0}$, $H^{0}$, $G^{1}$}, {A$\rightarrow$B, D$\rightarrow$G, F$\rightarrow$G, H$\rightarrow$G}) \newline

%         (B, C) \newline
%         data structure: ({$A^{0}$, $C^{0}$, $B^{1}$, $D^{0}$, $F^{0}$, $H^{0}$, $G^{1}$}, {A$\rightarrow$B, C$\rightarrow$B, D$\rightarrow$G, F$\rightarrow$G, H$\rightarrow$G}) \newline

%         (C, G) \newline
%         data structure: ({$A^{0}$, $C^{0}$, $B^{1}$, $D^{0}$, $F^{0}$, $H^{0}$, $G^{2}$}, {A$\rightarrow$B, C$\rightarrow$B, D$\rightarrow$G, F$\rightarrow$G, H$\rightarrow$G, B$\rightarrow$G}) \newline

%         (C, D) \newline
%         data structure: ({$A^{0}$, $C^{0}$, $D^{0}$, $F^{0}$, $H^{0}$, $B^{1}$, $G^{2}$}, {A$\rightarrow$B, C$\rightarrow$G, D$\rightarrow$G, F$\rightarrow$G, H$\rightarrow$G, B$\rightarrow$G}) \newline

%         (A, E) \newline
%         data structure: ({$A^{0}$, $C^{0}$, $D^{0}$, $F^{0}$, $H^{0}$, $E^{0}$, $B^{1}$, $G^{2}$}, {A$\rightarrow$B, C$\rightarrow$G, D$\rightarrow$G, F$\rightarrow$G, H$\rightarrow$G, B$\rightarrow$G, E$\rightarrow$G}) \newline

%         (D, H), (E, F), (B, F), (B, G), (A, F) \newline
%         data structure: ({$A^{0}$, $C^{0}$, $D^{0}$, $F^{0}$, $H^{0}$, $E^{0}$, $B^{1}$, $G^{2}$}, {A$\rightarrow$G, C$\rightarrow$G, D$\rightarrow$G, F$\rightarrow$G, H$\rightarrow$G, B$\rightarrow$G, E$\rightarrow$G}) \newline
%       \end{enumerate}
%     %5
%     \item 
%         Function is in file sum\_checker.py
    
%     %6
%     \item
%         Solver is in file KirkmanSolver.java
        
%     \end{enumerate}
% \end{document}








% %basic graph with nodes and arrows:
% \begin{tikzpicture}[->,>=stealth',shorten >=1pt,auto,node distance=2cm,
%     thin,main node/.style={circle,fill=blue!20,draw,font=\sffamily\Large\bfseries}]

% \node[main node] (1) {1};
% \node[main node] (2) [right of=1] {2};
% \node[main node] (3) [right of=2] {3};
% \node[main node] (4) [right of=3] {4};
% \node[main node] (5) [below of=1] {5};
% \node[main node] (6) [below of=2] {6};
% \node[main node] (7) [below of=3] {7};
% \node[main node] (8) [below of=4] {8};

% \path[every node/.style={font=\sffamily\small}]
% (1) edge [bend right] (2)
% (2) edge [bend right] (3)
% (1) edge [bend left=60] (8);

% \end{tikzpicture}