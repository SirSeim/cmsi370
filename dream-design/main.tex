\documentclass[journal,letterpaper]{article}

%Imports

\usepackage[utf8]{inputenc}
%headings
\usepackage{fancyhdr}
%double-space
\usepackage{setspace}
%variable margins
\usepackage{geometry}
%make the margins a bit less ridiculous
\newgeometry{top=1in,bottom=1.5in,left=1in,right=1in}
%images
\usepackage{graphicx}
\pagestyle{fancy}

\usepackage{tikz}
\usetikzlibrary{arrows}


\title{Dream Design - Better Apple Watch}
\author{Edward Seim}


\begin{document}
    \maketitle
    
    \section{Introduction}
    \label{Introduction}

    As an avid advocate of the Pebble Watch, the main thing that dissappoints me about the Apple Watch is the simplicity that is lost by bringing the complexity of iPhone apps to the watch screen. The root of this problem is the use of the touchscreen. As a Pebble user, which has only buttons to use the device with, it makes it possible to use the watch with minimal thought and once you've used it for a while, you can eve perform tasks without looking at it. My goal with my improved Apple Watch interface is to minimize use of the touchscreen to only when its eseential. 

    \section{System}
    \label{System}

    Without any change to the hardware, the Apple Watch has the neccesary buttons to emulate a Pebble type simplicity. using the Digital crown as a select and scrolling mechanism, and the side button as the back button, you have the same capabilities as the Pebble Watch, with the advantage of the touchscreen for specific uses and the better screen to display better eye candy.

    The most important change to the overall design of the interface is the change over to a menu type system. This means that the root level is the watch face. Hitting the Digital Crown will go into the main menu. Scrolling on this top menu will show the various different apps that you want use, for example the messaging or music apps. Hitting the back button will take you up a level in the menus. 

    \section{Conclusion}
    \label{Conclusion}

    Blah


\end{document}
