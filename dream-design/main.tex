\documentclass[journal,letterpaper]{article}

%Imports

\usepackage[utf8]{inputenc}
%headings
\usepackage{fancyhdr}
%double-space
\usepackage{setspace}
%variable margins
\usepackage{geometry}
%make the margins a bit less ridiculous
\newgeometry{top=1in,bottom=1.5in,left=1in,right=1in}
%images
\usepackage{graphicx}
\pagestyle{fancy}

\usepackage{tikz}
\usetikzlibrary{arrows}


\title{Dream Design - Better Apple Watch}
\author{Edward Seim}


\begin{document}
    \maketitle
    
    \section{Introduction}
    \label{Introduction}

    As an avid advocate of the Pebble Watch, the main thing that dissappoints me about the Apple Watch is the simplicity that is lost by bringing the complexity of iPhone apps to the watch screen. The root of this problem is the use of the touchscreen. As a Pebble user, which has only buttons to use the device with, it makes it possible to use the watch with minimal thought and once you've used it for a while, you can eve perform tasks without looking at it. My goal with my improved Apple Watch interface is to minimize use of the touchscreen to only when its eseential. 

    \section{Analysis}
    \label{Analysis}

    Blah

    \section{Conclusion}
    \label{Conclusion}

    Blah


\end{document}
